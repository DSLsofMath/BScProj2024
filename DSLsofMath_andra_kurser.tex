% Options for packages loaded elsewhere
\PassOptionsToPackage{unicode}{hyperref}
\PassOptionsToPackage{hyphens}{url}
%
\documentclass{article}
\usepackage{geometry}
\geometry{a4paper}
\usepackage{amsmath,amssymb}
\usepackage{lmodern}
\usepackage{iftex}
\usepackage[T1]{fontenc}
\usepackage[utf8]{inputenc}
\usepackage{textcomp} % provide euro and other symbols
\makeatletter
\@ifundefined{KOMAClassName}{% if non-KOMA class
  \IfFileExists{parskip.sty}{%
    \usepackage{parskip}
  }{% else
    \setlength{\parindent}{0pt}
    \setlength{\parskip}{6pt plus 2pt minus 1pt}}
}{% if KOMA class
  \KOMAoptions{parskip=half}}
\makeatother
\usepackage{xcolor}
\IfFileExists{xurl.sty}{\usepackage{xurl}}{} % add URL line breaks if available
\IfFileExists{bookmark.sty}{\usepackage{bookmark}}{\usepackage{hyperref}}
\hypersetup{
  hidelinks,
  pdfcreator={LaTeX via pandoc}}
\urlstyle{same} % disable monospaced font for URLs
\setlength{\emergencystretch}{3em} % prevent overfull lines
\providecommand{\tightlist}{%
  \setlength{\itemsep}{0pt}\setlength{\parskip}{0pt}}
\setcounter{secnumdepth}{-\maxdimen} % remove section numbering
\ifLuaTeX
  \usepackage{selnolig}  % disable illegal ligatures
\fi

\author{}
\date{}

\begin{document}
\section{Matematikens domänspecifika språk för andra kurser}\label{dslsofmath-matematikens-domuxe4nspecifika-spruxe5k-fuxf6r-andra-kurser}
[DSLsofMathBSc24]

Förslagslämnare: Patrik Jansson

\subsection{Bakgrund:}\label{bakgrund}

DSLsofMath {[}1,2,3{]} är namnet på ett pedagogiskt projekt som lett
till en valfri kurs i årskurs 2-3 riktad till datavetare och matematiker
på Chalmers och GU. Kursen presenterar klassiska matematiska ämnen från
ett datavetenskapligt perspektiv: genom att specificera de introducerade
begreppen, vara uppmärksam på syntax och typer, och slutligen genom att
bygga domänspecifika språk for vissa matematiska områden. (Exempelvis
linjär algebra, Laplace-transform, potensserier, derivator.)

Inspirerat av detta har flera studentgrupper genomfört
kandidatarbetesprojekt under de senaste åren med följande resultat:

\begin{itemize}
\tightlist
\item
  2016: Programmering som undervisningsverktyg för Transformer, signaler
  och system - Utvecklingen av läromaterialet TSS med DSL
\item
  2018: Ett komplementerande läromaterial för datastudenter som lär sig
  fysik - Läromaterialet Learn You a Physics for Great Good!
\item
  2020: A Computer Science Approach to Teaching Control Theory -
  Developing Learning Material Using Domain-Specific Languages
\item
  2022: HasLin - ett DSL för linjär algebra - Utvecklandet av ett
  matematiskt domänspecifikt språk för linjär algebra i Haskell
\end{itemize}

\subsection{Projektbeskriving:}\label{projektbeskriving}

Det här kandidatprojektet går ut på att ta fram DSLsofMath-inspirerat
kompletterande material för andra närliggande kurser som exempelvis

\begin{itemize}
\tightlist
\item
  Matematisk statistik och diskret matematik, eller
\item
  Flervariabelanalys, eller
\item
  Data Science \& AI, eller
\item
  andra kurser som ni känner skulle må bra av mer fokus på syntax, typer
  och funktioner.
\end{itemize}

Implementationsspråk är Haskell och Agda och målet är dels att förbättra
förståelsen hos projektmedlemmarna av de kurser och ämnen som väljs och
dels att ge framtida studenter mer material att arbeta med. Materialet
som utvecklas skall finnas öppet tillgängligt på github.

Efter ett par år med huvudfokus på lärmaterial (tutorial / lecture
notes) är fokus i år mer inriktat mot korrekthet: DSL, typer,
specifikation, test, bevis.

Att göra (``produkt''):

\begin{itemize}
\tightlist
\item
  Designa och implementera (ett par) DSL för det valda området
\item
  Specificera lagar som bör gälla i Haskell eller Agda
\item
  Testa de lagar som kan testas med QuickCheck
\item
  Bevisa någon eller några lagar i Agda
\item
  \ldots{} samt dokumentation i form av kandidatarbetesrapport mm.
\end{itemize}

\subsection{Länkar:}\label{luxe4nkar}

\begin{enumerate}
\def\labelenumi{\arabic{enumi}.}
\tightlist
\item \href{https://github.com/DSLsofMath/DSLsofMath}{github.com/DSLsofMath/DSLsofMath}
\item \href{https://www.cse.chalmers.se/~patrikj/papers/Ionescu_Jansson_DSLsofMath_TFPIE_2015_paper_preprint.pdf}{DSLsofMath\_TFPIE\_2015\_paper\_preprint.pdf}
\item \href{https://www.cse.chalmers.se/~patrikj/papers/Janssonetal_DSLsofMathCourseExamplesResults_preprint_2018-08-17.pdf}{DSLsofMathCourseExamplesResults\_preprint\_2018-08-17.pdf}
\end{enumerate}

\subsection{Målgrupp:}\label{muxe5lgrupp}

DV, D, IT, TM

\subsection{Särskilda förkunskaper:}\label{suxe4rskilda-fuxf6rkunskaper}

Funktionell programmering (Haskell) och kursen DSLsofMath eller gott om
matematik (TM-programmet eller liknande).

(Det kan gå att ta kursen DSLsofMath parallellt med projektet, men det
blir svårare.)

\subsection{Handledare:}\label{handledare}

Patrik Jansson eller annan person inom FP-gruppen.

\end{document}
